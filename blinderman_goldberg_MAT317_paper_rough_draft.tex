\documentclass[12pt]{article}
\usepackage{amsmath, amssymb, amsthm}    
\usepackage{parskip}  
\usepackage{enumerate}  
\usepackage{geometry}    
\usepackage{titling}     
\usepackage{color}       
\usepackage{hyperref}   
\usepackage{tipa}
\usepackage{graphicx}     
\usepackage{ mathrsfs }

%\geometry{legalpaper, portrait, margin=1in}
%\linespread{2}
\setlength{\parindent}{1.5cm}
\setlength{\droptitle}{-5em}

\newcommand{\N}{\mathbb{N}}
\newcommand{\Z}{\mathbb{Z}}
\newcommand{\R}{\mathbb{R}}
\newcommand{\C}{\mathbb{C}}
\newcommand{\fc}{F^{\C}}

\newcommand{\poincare}{Poincar\'{e} }
\newcommand{\Tr}{\text{Tr}}
\newcommand{\Range}{\text{Range}}
\newcommand{\Ker}{\text{Ker}}
\newcommand{\inv}{^{-1}}
\newcommand{\Log}{\text{Log}}

\newcommand{\spceq}{\hspace{2mm} = \hspace{2mm}}
\newcommand{\ttspc}{\hspace{1mm}}
\newcommand{\ttc}{, \hspace{1mm}}


\newcommand{\lftmat}[4]{\begin{bmatrix} {#1} & {#2} \\ {#3} & {#4} \end{bmatrix}}
\newcommand{\stanlftmat}{\lftmat{a}{b}{c}{d}}
\newcommand{\pointmat}[2]{\lftmat{{#2}}{{#1}}{-1}{-{#2}}}
\newcommand{\stanpointmat}{\pointmat{x}{y}}
\newcommand{\linenoendmat}[2]{\begin{bmatrix} -{#2} & -{#1} \\ 1 & {#2} \end{bmatrix}}
\newcommand{\stanlinenoendmat}{\linenoendmat{l_1}{l_2}}
\newcommand{\lineendmat}[2]{\begin{bmatrix} -1 & -{#1} \\ 0 & 1 \end{bmatrix}}
\newcommand{\stanlineendmat}{\lineendmat{l_1}{l_2}}

\newcommand{\specialend}{(\infty^2\ttc\infty)}

% Theorem counters
\theoremstyle{plain}
\newtheorem{theorem}{Theorem}[section]
\newtheorem{definition}{Definition}[section]
\newtheorem{corollary}[theorem]{Corollary}
\newtheorem{lemma}{Lemma}[section]
\newtheorem{proposition}[theorem]{Proposition}

% Problem environments
\theoremstyle{definition}
\newtheorem{problem}[theorem]{Problem}
\newtheorem{example}[definition]{Example}

\title{The Hyperbolic Geometry of Linear Fractional Transformations in Complexified Fields}
\author{Anna Blinderman, Logan Goldberg}
\date{}

\begin{document}

\maketitle


Hyperbolic geometry is a model of geometry resulting from the assumption of the first four postulates of Euclid (the ability to draw a straight line segment by joining any two points, the ability to extend any line segment indefinitely to a straight line, the ability to take a line segment and draw a circle with that segment as radius and one end as center, and the congruency of all right angles) and the negation of the fifth \href{http://mathworld.wolfram.com/EuclidsPostulates.html}{[source]}. The fifth postulate states that ``if a straight line, which intersects two straight lines, form interior angles on the same side, smaller than two right angles, then these straight lines, extended to infinite, will intersect on the side where the interior angles are less than two right angles." [\href{https://pdfs.semanticscholar.org/ed0e/d1fee9bbe60b24be373ac1207d17ecb90b4a.pdf}{source}]. In hyperbolic geometry, this postulate is replaced with its negation, ``Through an exterior point to a straight line we can construct an infinite number of parallels to that straight line." \href{https://pdfs.semanticscholar.org/ed0e/d1fee9bbe60b24be373ac1207d17ecb90b4a.pdf}{[source]}.
	
Henri \poincare put forth two models for hyperbolic geometry: the \poincare half-plane and the \poincare disk. We focus here on the \poincare disk model. The set of points in the model are given by $D = \{(x\ttc y): x^2 + y^2 < 1\}$ \href{http://math2.uncc.edu/~frothe/3181alllhyp1_7.pdf}{[source]}.  \footnote{Put more stuff here once we decide which figures to include.} 
	
We will first set forth the notation we will use throughout our paper. In general, we will denote the coordinates of points corresponding to point reflections as $(x \ttc y) \ttc (x' \ttc y')$, etc. As discussed later, such points will be in the interior of our \poincare disk. On the other hand, points defining lines (which lie outside of the \poincare disk) will be denoted $(l_1 \ttc l_2) \ttc (l_1' \ttc l_2')$, etc. We will then discuss general linear fractional transformations, point reflections, and two types of line reflections both algebraically and with the aid of the \poincare disk model. We first begin with an arbitrary field $F$ with the property that every positive $x \in F$ has a square root. We define the group of hyperbolic transformations over $F$ to be those linear fractional transformations\footnote{Might want to show this is a group??} given by
\begin{equation} 
	z \mapsto \frac{az+b}{cz+d} \text{ where } z \in \C, \hspace{2mm} a \ttc b \ttc c \ttc d \in F \text{ and } ad-bc \neq 0. 
\end{equation}
For convenience, we will encode our transformations of the above form into matrices\footnote{Maybe make this into a set?} 
\begin{equation}
	\lambda \stanlftmat \text{ where } \lambda \ttc a \ttc b \ttc c \ttc d \in F \text{ and } \lambda \neq 0. 
\end{equation}
It should be explicitly noted that these matrices are used only to encode our transformations\footnote{We should make an appendix of off-hand proofs like this isomorphism}. That is, to ``apply" a transformation encoded the matrix in (2) to a point $z_0 \in \C$, we mean that we will apply a corresponding linear fractional transformation $f$ defined by (1) to the point $z_0$. Because all linear fractional transformations we will use are of the above form\footnote{Which form? 2?}, whenever we say ``matrix" we are referring to a $2 \times 2$ matrix. Finally, later we will be associating points with point reflections and lines with line reflections. In order to distinguish these geometric objects from their corresponding reflections, we will write $f_P(P_0)$ to mean the point reflection of the point $P_0$ over a point $P$ via the point reflection $f_P$ and $f_l(P_0)$ to be the line reflection of point $P_0$ over the line $l$ via the line reflection $f_l$\footnote{We might want to make note somewhere around here about the uniqueness of points to their point reflections and similarly for line reflections}. In a similar vein, we define $M_P$ and $M_l$ to be a matrix representation of a point reflection about $P$ and a line reflection about $l$ respectively\footnote{I think the solution here is to just be careful about when we are talking about the items we're talking about and use the specific $M$ or $f$ or actual elements in the correct context}.

We now go forth to discuss hyperbolic point reflections and hyperbolic line reflections. As in Euclidean geometry, both of these transformations are of order two. The matrices encoding point reflections have positive determinant; those encoding line reflections have negative determinant (recall by (1) that no linear fractional transformation can be encoded in a matrix with determinant 0 since linear fractional transformations with $ad - bc = 0$ do not have inverses). 

Suppose that $M$ is a matrix encoding some linear fractional transformation of order two. We must have that $M^2 = \lambda I$ for some constant $\lambda \in F$ \footnote{$\lambda I$ is saying we can pick any representative of the coset $\lambda I$ and still get the identity, so $\lambda$ is a way of picking which representative. We can also think of $\lambda$ as $\lftmat{\lambda}{0}{0}{\lambda}$ to encode $\lambda$ as a matrix and make these cosets more clear, maybe call it $M_\lambda$?}, where $I$ is the usual $2 \times 2$ identity matrix. In this case, we want to have that 
	\[
		\stanlftmat \stanlftmat = \lambda \lftmat{1}{0}{0}{1}
	\]
for some $\lambda \in F$. Expanding this further, we see that we must have 
	\[
		\lftmat{a^2 + bc}{b(a+d)}{c(a+d)}{bc+d^2} =  \lftmat{\lambda}{0}{0}{\lambda}
	\]
The only way this can be true \footnote{barring the trivial case in which $M$ is already of the form $\lambda I$} is if $\Tr(M) = 0$. Let $(x \ttc y)$ be a point in the \poincare disk. Then, for the matrix encoding the hyperbolic point reflection over $(x \ttc y)$ to satisfy $\Tr(M)$ and $\det M = -(y^2 - x) > 0$, we must have that $M$ is of the form \footnote{Is it strictly necessary that we have this form or is it just an easy classification mechanism?} 
\begin{equation} 
	\lambda \stanpointmat \text{ where } \lambda \in F \text{ and } x > y^2. 
\end{equation}	
We will associate each point reflection with a unique point in the disk\footnote{So apparently we're just saying this...?}. It follows that our points are given uniquely by $(x \ttc y) \in F^2$ inside the convex domain of the parabola $x = y^2$.

To be clear, we are in no way mapping this parabola to the unit disk. Similarly, we are neither mapping the points within the parabola to those in the \poincare disk nor those outside the parabola to outside the disk. Rather, we are simply classifying the points in $F^2$ into three disjoint subsets: those that will be in the interior of the \poincare disk (those in the convex domain defined by the parabola $x = y^2$, those that will be on the \poincare disk itself (those on the parabola $x = y^2$), and those that will be outside of the \poincare disk (those neither inside nor on the parabola $x = y^2$). 

Next \footnote{I guess we're just adding ends so we can have two kinds of lines? Why doesn't Guggenheimer \textit{ever} give motivation?} we define $F_*$ to be the extension of $F$ by $\infty$ as usual \footnote{We should definitely elaborate on this point rather than just say ``as usual". More appendix!}. We define\footnote{is ``define" the right word here?} an end of our geometry to be the points on our parabola, namely those pairs $(x \ttc y) \in F_*^2$ where $y = x^2$. The end $\specialend$ is of particular interest to us in our classification of line reflections. In particular, the reflection over a line $l$ given by $(l_1\ttc l_2)$ \footnote{I still don't understand how a point defines a line, but I think it has to do with the whole thing conjugacy below and will become clearer once we plot more.} that does not pass through the end $\specialend$ can be encoded in the matrix
\[\stanlinenoendmat\]
If the line $l$ given by $(l_1\ttc l_2)$ does pass through the end $\specialend$, then the reflection over $l$ can be given by the following matrix\footnote{Presumably there will be intuition for this as well once we generate our figures.}:
\[
	\stanlineendmat
\]

Just as we uniquely associated all points with their reflections, we will uniquely associate all lines $l$ with their reflections. \footnote{What about scalar multiples? I tried to work it out on 11/14 and I think it's impossible  but I want to make sure these are actually unique.} 

Notice that if a point $P$ is on a line $l$, then the reflection of $P$ over $l$ will have no effect. Regarding our point $P$ and line $l$ as their respective reflections, we then have that $f_l(P) = P$. For the same reason \footnote{presumably...}, we have that $M_l M_P M_l\inv = M_P$, whence $M_l M_P = M_P M_l$. We will first consider this in terms of matrices for lines not passing through the end $\specialend$. We have the following\footnote{$M_l M_P$ is a representative of $M_P M_l$, hence the $\lambda$}:
\begin{equation} 
\stanlinenoendmat\stanpointmat \spceq \lambda \stanpointmat\stanlinenoendmat
\end{equation}	
Expanding, we see that we must have
	\[
		\lftmat{-l_2y+l_1}{-l_2x+l_1y}{y-l_2}{x-l_2y} \spceq \lambda \lftmat{-l_2y+x}{-l_1y+l_2x}{l_2-y}{l_1-l_2y}
	\]
Looking at the lower-left corner of our matrices, we see that we must have $\lambda = -1$ since $y - l_2 = -1(l_2 - y)$. If we then consider the upper-left corner, we must have that $-l_2y + l_1 = -1(-l_2y + x)$. Simplifying, we see that equations of lines not passing through the end $\specialend$ must be of the form
\begin{equation}
	x - 2l_2y + l_1  = 0
\end{equation} 
Now, if $l$ is a line passing through the end $\specialend$ and $P$ is some point on $l$, we again have that $M_l M_P = M_P M_l$. Again realizing this in terms of matrices, we have that 
\begin{equation} 
	\stanlineendmat\stanpointmat \spceq \lambda \stanpointmat\stanlineendmat
\end{equation}	
We again expand to see that 
	\[
		\lftmat{-y+l_1}{-x+l_1y}{-1}{-y} \spceq \lambda \lftmat{-y}{-l_1y+x}{1}{l_1-y}
	\]
By considering the lower-left entry into these matrices we see that $\lambda = -1$. We can now find the general form of such lines by seeing from the upper-left entries that $-y+l_1 = -y$. Simplified, our form is 
\begin{equation}
	2y = l_1
\end{equation} 

%%%%%%%%%%%%%%%%%%%%%%%%%%% VI.preA %%%%%%%%%%%%%%%%%%%%%%%%%%%

Recall that in order for $(F,+,\cdot)$ to be a field, it must satisfy the following axioms.
\begin{enumerate}
	\item $+$ must be associative and commutative.
	\item There exists an additive identity in $F$.
	\item Each $a\in F$ has an additive inverse.
	\item $\cdot$ must be associative and commutative.
	\item $+$ and $\cdot$ satisfy the distributive properties.
	\item There exists an multiplicative identity in $F$.
	\item Each $a\in F$ has an multiplicative inverse.
\end{enumerate}
Previously, all of the algebraic manipulations were carried out over a real ordered field in which all positive numbers have square roots. However, we may also carry out similar calculations over a complexified field. Let $(F,+,\cdot)$ be a field in which $-1$, the additive inverse of $1$, is not the square of any number in $F$. We define $F^\C = F\times F$. Further we define two new binary operations $\oplus$ and $\odot$ such that for all $(a,b)\ttc (c,d)\in F^\C$,
	\[
		(a,b)\oplus(c,d) = (a + c\ttc b + d)
	\]
and
	\[
		(a,b)\odot(c,d) = (ac - bd\ttc ad + bc).
	\]
wherein $+,-,\cdot$ are from $F$. We call $(F^\C,\oplus,\odot)$ the complexification of $F$ and claim that it is a field. 

First we will show that $\oplus$ is associative. Let $(a,b)\ttc(c,d)\ttc(e,f)\in F^\C$.
\begin{align*}
	((a,b)\oplus(c,d))\oplus(e,f) & = (a + c\ttc b + d) + (e,f)\\
	& = ((a + c) + e\ttc (b + d) + f)\\
	& = (a + (c + e)\ttc b + (d + f)) & \text{(associativity of $+$)}\\
	& = (a,b) \oplus (c + e\ttc d + f)\\
	& = (a,b)\oplus ((c,d) \oplus (e,f))
\end{align*}
Hence $\oplus$ is associative. Now we argue that $\oplus$ is commutative. Let $(a,b)\ttc(c,d)\in F^\C$.
\begin{align*}
	(a,b)\oplus (c,d) & = (a + c\ttc b + d)\\
	& = (c + a\ttc d + b) & \text{(commutativity of $\oplus$)}\\
	& = (c,d)\oplus (a,b)
\end{align*}
Hence $\oplus$ is commutative. Now we argue that $F$ has an additive identity. Let $(a,b)\in F$ be arbitrary. We claim that $(0,0)\in F^\C$ is the additive identity of $F$. Using the fact that $0\in F$ is the additive identity, we obtain
	\[
		(a,b)\oplus (0,0) = (a + 0\ttc b + 0) = (a,b) = (0 + a\ttc, 0 + b) = (0,0) \oplus (a,b).
	\]
Therefore $(0,0)\in F^\C$ is the additive identity. Finally we argue that $F^\C$ has additive inverses. Let $(a,b)\in F^\C$ be arbitrary. We claim that $(-a,-b)\in F^\C$ is the additive inverse of $(a,b)$. Note that since $F$ is a field, $-a\ttc -b$ exist and so $(-a,-b)\in F^\C$. Furthermore,
	\[
		(a,b)\oplus (-a,-b) = (a + -a\ttc b + -b) = (0,0) = (-a + a\ttc, -b + b) = (-a,-b) \oplus (a,b).
	\]
Hence $\oplus$ behaves like field addition. 

	Now we argue via a similar line of reasoning that $\odot$ satisfies the properties of field multiplication. First we claim that $\odot$ is associative. Let $(a,b)\ttc(c,d)\ttc(e,f)\in F^\C$ be arbitrary. Using distributivity in $F$ and associativity and commutativity of $\oplus$ and $\odot$, we obtain
\begin{align*}
		((a,b)\odot(c,d))\odot(e,f) & = (ac - bd\ttc ad + bc)\odot (e,f)\\
		& = ((ac - bd)e - (ad + bc)f\ttc (ac - bd)f + (ad + bc)e)\\
		& = (ace - bde - adf - bcf\ttc acf - bdf + ade + bce)\\
		& = (a(ce - df) - b(de + cf)\ttc b(ce - df) + a(de + cf))\\
		& = (a,b)\odot (ce - df\ttc de + cf)\\
		& = (a,b)\odot ((c,d)\odot (e,f))
\end{align*}
Hence $\odot$ is associative. Now we show that $\odot$ is commutative. Let $(a,b)\ttc(c,d)\in F^\C$ be arbitrary. Then
	\begin{align*}
		(a,b)\odot (c,d) & = (ac - bd\ttc ad + bc)\\
		& = (ca - db\ttc da + cb) & \text{(commutativity of $\cdot$)}\\
		& = (c,d)\odot (a,b)
	\end{align*}
	Therefore $\odot$ is commutative. Next we show that $F^\C$ has the distributive property. Let $(a,b)\ttc(c,d)\ttc(e,f)\in F^\C$ be arbitrary. Since $\odot$ is commutative, it suffices to argue that
	\[
		(a,b)\odot((c,d)\oplus(e,f)) = (a,b)\odot(c,d)\oplus(a,b)\odot(e,f).
	\]
	Following previous arguments, we find that
	\begin{align*}
		(a,b)\odot((c,d)\oplus(e,f)) & = (a,b)\odot(c + e\ttc d + f)\\
		& = (a(c + e) - b(d + f)\ttc a(d + f) + b(c + e))\\
		& = (ac + ae - bd - bf\ttc ad + af + bc + be) & \intertext{(distributivity in $F^\C$)}
		& = ((ac - bd) + (ae - bf)\ttc (ad + bc) + (af + be)) & \intertext{(commutativity and associativity of $+$)}
		& = (ac - bd\ttc ad + bc)\oplus (ae - bf\ttc af + be)\\
		& = (a,b)\odot (c,d)\oplus (a,b)\odot(e,f)
	\end{align*}
	Hence $F^\C$ has the distributive property. Next we show that $F^\C$ has a multiplicative identity. We claim that $(1,0)$ is the multiplicative identity of $F^\C$. Let $(a,b)\in F^\C$ be arbitrary. Note that since $F$ is a field, $1\in F$ is the multiplicative identity. Then
	\[
		(a,b)\odot(1,0) = (a\cdot 1 - b\cdot 0\ttc a\cdot 0 + b\cdot 1) = (a,b) = (1\cdot a - 0\cdot b\ttc 0\cdot a + 1\cdot b) = (1,0)\cdot (a,b).
	\]
	Hence $F^\C$ has a multiplicative identity. Finally we claim that $F^\C$ has multiplicative inverses. Suppose $(a,b)\in F^\C\backslash\{(0,0)\}$. We hope to find $(x,y)\in F^\C$ with $(a,b)\odot(x,y) = (1,0)$. Since $\odot$ is commutative, this result holds for $(x,y)\odot(a,b)$ as well. Performing the multiplication, we find that
	\[
		(a,b)\odot(x,y) = (ax - by\ttc ay + bx) = (1,0)
	\]
	and so we obtain the equations
	\begin{align*}
		ax - by & = 1\\
		ay + bx & = 0
	\end{align*}
	Now we need to solve for $x$ and $y$. Multiply the upper equation by $a$ and the lower equation by $b$ to obtain
	\begin{align*}
		a^2x - aby & = a\\
		aby - b^2x & = 0
	\end{align*}
	Adding the two equations together, we find that
	\[
		x = \frac{a}{a^2 + b^2}.
	\]
	Via a similar process of multiplying the first equation by $b$ and the second equation by $a$ and adding the results, we also obtain
	\[
		y = \frac{-b}{a^2 + b^2}.
	\]
	Now we hope that there is no $(a,b)\in F^\C\backslash\{(0,0)\}$ with $a^2 + b^2 = 0$. Suppose such values do exist however. Since $(a,b)\neq (0,0)$, suppose $a\neq 0$. Then
	\begin{align*}
		a^2 + b^2 = 0 & \Rightarrow b^2 = -a^2\\
		& \Rightarrow \frac{b^2}{a^2} = -1\\
		& \Rightarrow \left(\frac{b}{a}\right)^2 = -1.
	\end{align*}
 	In other words, $-1$ is the square of $\frac{b}{a}\in F$, a contradiction since we assumed that no element in $F$ had $-1$ as its square. A similar argument can be made for when $b \neq 0$ using the fraction $\frac{a}{b}$. Hence $a^2 + b^2 \neq 0$ for any $(a,b)\in F^\C\backslash\{(0,0)\}$. Therefore $F^\C$ has multiplicative inverses. Therefore $(F^\C,\oplus,\odot)$ is a field. 

%%%%%%%%%%%%%%%%%%%%%%%%%%% I.A %%%%%%%%%%%%%%%%%%%%%%%%%%%

Define the group
	\[
		A = \left\lbrace z\mapsto \frac{az + b}{cz + d}\colon a,b,c,d\in F\text{ with } ad - bc \neq 0\right\rbrace
	\]
of linear fractional transformations and let $GL_2(F)$ denote the group of invertible $2\times 2$ matrices with entries in $F$. Also define the subset
	\[
		F^* = \left\lbrace\lftmat{\lambda}{0}{0}{\lambda}\colon \lambda\in F\text{ with } \lambda\neq 0 \right\rbrace.
	\]
	We argue that $F^*$ is a normal subgroup of $GL_2(F)$. \footnote{Insert math here} With this in mind, we now define the map $\varphi\colon A\rightarrow GL_2(F)/F^*$ by
	\[
		\varphi\left(z\mapsto \frac{az + b}{cz + d}\right) = \stanlftmat F^*
	\]
	We claim that $\varphi$ is a homomorphism.  \footnote{Insert more math here}Now that we've established that $\varphi$ is a homomorphism, we argue that $\varphi$ is surjective.  \footnote{And a little bit more right here}  Next we identify the kernel of $\varphi$ in order to establish an isomorphism between $A/\ker(\varphi)$ and $GL_2(F)/F^*$\footnote{Technically we don't need the kernel to identify the isomorphism; it's just noteworthy and tells us when we can't recover a linear fractional transformation.}. Finally by the First Isomorphism Theorem, we conclude that $A/\ker(\varphi)$ \footnote{insert whatever the kernel turns out to be here} is isomorphic to $GL_2(F)/F^*$. 
	







\end{document}
