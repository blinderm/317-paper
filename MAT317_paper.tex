\documentclass[12pt]{article}
\usepackage{amsmath, amssymb, amsthm}    
\usepackage{parskip}  
\usepackage{enumerate}  
\usepackage{geometry}    
\usepackage{titling}     
\usepackage{color}       
\usepackage{hyperref}   
\usepackage{tipa}
\usepackage{graphicx}     
\usepackage{ mathrsfs }

%\geometry{legalpaper, portrait, margin=1in}
%\linespread{2}
\setlength{\parindent}{1.5cm}   
\setlength{\droptitle}{-5em}

\newcommand{\N}{\mathbb{N}}
\newcommand{\Z}{\mathbb{Z}}
\newcommand{\R}{\mathbb{R}}
\newcommand{\C}{\mathbb{C}}
\newcommand{\fc}{F^{\C}}

\newcommand{\poincare}{Poincar\'{e} }
\newcommand{\Tr}{\text{Tr}}
\newcommand{\Range}{\text{Range}}
\newcommand{\Ker}{\text{Ker}}
\newcommand{\inv}{^{-1}}
\newcommand{\Log}{\text{Log}}

\newcommand{\spceq}{\hspace{2mm} = \hspace{2mm}}
\newcommand{\ttspc}{\hspace{1mm}}
\newcommand{\ttc}{, \hspace{1mm}}


\newcommand{\lftmat}[4]{\begin{bmatrix} {#1} & {#2} \\ {#3} & {#4} \end{bmatrix}}
\newcommand{\stanlftmat}{\lftmat{a}{b}{c}{d}}
\newcommand{\pointmat}[2]{\lftmat{{#2}}{{#1}}{-1}{-{#2}}}
\newcommand{\stanpointmat}{\pointmat{x}{y}}
\newcommand{\linenoendmat}[2]{\begin{bmatrix} -{#2} & -{#1} \\ 1 & {#2} \end{bmatrix}}
\newcommand{\stanlinenoendmat}{\linenoendmat{l_1}{l_2}}
\newcommand{\lineendmat}[2]{\begin{bmatrix} -1 & -{#1} \\ 0 & 1 \end{bmatrix}}
\newcommand{\stanlineendmat}{\lineendmat{l_1}{l_2}}

\newcommand{\specialend}{(\infty^2\ttc\infty)}


\title{Fixing Heinrich Guggenheimer's Fucking Wild Notation and Generally Shitty/Vague Paper}
\author{Anna Blinderman, Logan Goldberg}
\date{}

\begin{document}

\maketitle


Hyperbolic geometry is a model of geometry resulting from the assumption of the first four postulates of Euclid (the ability to draw a straight line segment by joining any two points, the ability to extend any line segment indefinitely to a straight line, the ability to take a line segment and draw a circle with that segment as radius and one end as center, and the congruency of all right angles) and the negation of the fifth [pretty copied from \href{http://mathworld.wolfram.com/EuclidsPostulates.html}{[here]}]. The fifth postulate states that ``if a straight line, which intersects two straight lines, form interior angles on the same side, smaller than two right angles, then these straight lines, extended to infinite, will intersect on the side where the interior angles are less than two right angles." [definitely copied from [\href{https://pdfs.semanticscholar.org/ed0e/d1fee9bbe60b24be373ac1207d17ecb90b4a.pdf}{here}]]. In hyperbolic geometry, this postulate is replaced with its negation, ``Through an exterior point to a straight line we can construct an infinite number of parallels to that straight line." [also copied from \href{https://pdfs.semanticscholar.org/ed0e/d1fee9bbe60b24be373ac1207d17ecb90b4a.pdf}{[here]}]. 
	
Henri \poincare put forth two models for hyperbolic geometry: the \poincare half-plane and the \poincare disk. We focus here on the \poincare disk model. The set of points in the model are given by $D = \{(x\ttc y): x^2 + y^2 < 1\}$. %http://math2.uncc.edu/~frothe/3181alllhyp1_7.pdf
	
We will first set forth the notation we will use throughout our paper. In general, we will denote the coordinates of points corresponding to point reflections as $(x \ttc y) \ttc (x' \ttc y')$, etc. As discussed later, such points will be in the interior of our \poincare disk. On the other hand, points defining lines (which lie outside of the \poincare disk) will be denoted $(l_1 \ttc l_2) \ttc (l_1' \ttc l_2')$, etc. We will then discuss general linear fractional transformations, point reflections, and two types of line reflections both algebraically and with the aid of the \poincare disk model. We first begin with an arbitrary field $F$ with the property that every positive $x \in F$ has a square root. We define the group of hyperbolic transformations over $F$ to be those linear fractional transformations given by
\begin{equation} 
z \mapsto \frac{az+b}{cz+d} \text{ where } z \in \C, \hspace{2mm} a \ttc b \ttc c \ttc d \in F \text{ and } ad-bc \neq 0. 
\end{equation}
For convenience, we will encode our transformations of the above form into matrices 
\begin{equation} 
\lambda \stanlftmat \text{ where } \lambda \ttc a \ttc b \ttc c \ttc d \in F \text{ and } \lambda \neq 0. 
\end{equation}
It should be explicitly noted that these matrices are used only to encode our transformations. That is, to ``apply" a transformation encoded the matrix in (2) to a point $z_0 \in \C$, we mean that we will apply the function $f: \C \rightarrow \C$ [HOW DO WE ENSURE IT'S $\C$? DO WE?] given by $f(z) = \frac{az+b}{cz+d}$ to the point $z_0$. Because all linear fractional transformations we will use are of the above form, whenever we say ``matrix" we are referring to a $2 \times 2$ matrix. Finally, later we will be associating points with point reflections and lines with line reflections. In order to distinguish these geometric objects from their corresponding functions, we will write  $f_P(P_0)$ to mean the point reflection of the point $P_0$ over a point $P$ and $f_l(P_0)$ to be the line reflection of point $P_0$ over the line $l$.

We now go forth to discuss hyperbolic point reflections and hyperbolic line reflections. As in Euclidean geometry, both of these transformations are of order two. The matrices encoding point reflections have positive determinant; those encoding line reflections have negative determinant (recall by (1) that no linear fractional transformation can be encoded in a matrix with determinant 0). WHY THOUGH? THE STUFF WITH DETERMINANTS LIKE WHY IS IT TRUE

Suppose that $M$ is a matrix encoding some linear fractional transformation of order two. We must have that $M^2 = \lambda I$ for some constant $\lambda \in F$ HOW DO WE JUSTIFY THE EXISTENCE OF THE CONSTANT, where $I$ is the usual $2 \times 2$ identity matrix. In this case, we want to have that 
\[\stanlftmat \stanlftmat = \lambda \lftmat{1}{0}{0}{1}\]
for some $\lambda \in F$. Expanding this further, we see that we must have 
\[\lftmat{a^2 + bc}{b(a+d)}{c(a+d)}{bc+d^2} =  \lftmat{\lambda}{0}{0}{\lambda}\]
The only way this can be true (BARRING THE TRIVIAL CASE IN WHICH $M$ IS ALREADY OF THE FORM $\lambda I$) is if $\Tr(M) = 0$. Let $(x \ttc y)$ be a point in the \poincare disk. Then, for the matrix encoding the hyperbolic point reflection over $(x \ttc y)$ to satisfy $\Tr(M)$ and $\det M = -(y^2 - x) > 0$, we must have that $M$ is of the form I MEAN, DO WE REALLY *NEED* TO HAVE IT OF THIS FORM? 
\begin{equation} 
\lambda \stanpointmat \text{ where } \lambda \in F \text{ and } x > y^2. 
\end{equation}	
SO APPARENTLY WE'RE JUST SAYING THAT we will associate each point reflection with a unique point in the disk. It follows that our points are given uniquely by $(x \ttc y) \in F^2$ inside the convex domain of the parabola $x = y^2$.

To be clear, we are in no way mapping this parabola to the unit disk. Similarly, we are neither mapping the points within the parabola to those in the \poincare disk nor those outside the parabola to outside the disk. Rather, we are simply classifying the points from $F$ into three groups: those that will be in the interior of the \poincare disk (those in the convex domain defined by the parabola $x = y^2$, those that will be on the \poincare disk itself (those on the parabola $x = y^2$), and those that will be outside of the \poincare disk (those neither inside nor on the parabola $x = y^2$). 

Next I GUESS WE'RE JUST ADDING ENDS SO WE CAN HAVE TWO KINDS OF LINES? Well anyway, we next define $F_*$ to be the extension of $F$ by $\infty$ as usual CAN WE ACTUALLY JSUT SAY ``AS USUAL." We DEFINE?? an end of our geometry to be the points on our parabola, namely those pairs $(x \ttc y) \in F_*^2$ where $y = x^2$. The end $\specialend$ is of particular interest to us in our classification of line reflections. In particular, the reflection over a line $l$ given by $(l_1\ttc l_2)$ I STILL DON'T UNDERSTAND HOW A POINT DEFINES A LINE that does not pass through the end $\specialend$ can be encoded in the matrix
\[\stanlinenoendmat\]
If the line $l$ given by $(l_1\ttc l_2)$ does pass through the end $\specialend$, then the reflection over $l$ can be given by the following matrix:
\[\stanlineendmat\]
INTUITION FOR WHY THIS IS TRUE?

Just as we uniquely associated all points with their reflections, we will uniquely associate all lines $l$ with their reflections. WHAT ABOUT SCALAR MULTIPLES? I TRIED TO WORK IT OUT BUT I'M IFFY ABOUT THE SUPPOSED UNIQUENESS HERE. 

Notice that if a point $P$ is on a line $l$, then the reflection of $P$ over $l$ will have no effect. Regarding our point $P$ and line $l$ as their respective reflections, we then have that $lP = P$. FOR THE SAME REAONS I GUESS, we have that $lPl\inv = P$, whence $lP = Pl$. We will first consider this in terms of matrices for lines not passing through the end $\specialend$. We have the following:
\begin{equation} 
\stanlinenoendmat\stanpointmat \spceq \lambda \stanpointmat\stanlinenoendmat
\end{equation}	
Expanding, we see that we must have
\[
\lftmat{-l_2y+l_1}{-l_2x+l_1y}{y-l_2}{x-l_2y} \spceq \lambda \lftmat{-l_2y+x}{-l_1y+l_2x}{l_2-y}{l_1-l_2y}
\]
Looking at the lower-left corner of our matrices, we see that we must have $\lambda = -1$ since $y - l_2 = -1(l_2 - y)$. If we then consider the upper-left corner, we must have that $-l_2y + l_1 = -1(-l_2y + x)$. Simplifying, we see that equations of lines not passing through the end $\specialend$ must be of the form
\begin{equation}
x - 2l_2y + l_1  = 0
\end{equation} 
Now, if $l$ is a line passing through the end $\specialend$ and $P$ is some point on $l$, we again have that $lP = Pl$. Again realizing this in terms of matrices, we have that 
\begin{equation} 
\stanlineendmat\stanpointmat \spceq \lambda \stanpointmat\stanlineendmat
\end{equation}	
We again expand to see that 
\[
\lftmat{-y+l_1}{-x+l_1y}{-1}{-y} \spceq \lambda \lftmat{-y}{-l_1y+x}{1}{l_1-y}
\]
By considering the lower-left entry into these matrices we see that $\lambda = -1$. We can now find the general form of such lines by seeing from the upper-left entries that $-y+l_1 = -y$. Simplified, our form is 
\begin{equation}
2y=l_1
\end{equation} 

Recall that we did everything up there in an arbitrary field $F$. 

However, a similar process could be carried out 






\[\]
\newpage\begin{enumerate}[I.]
	\item Somewhere
	\begin{enumerate}[A.]
		\item $U/\Ker(\phi) \cong \Range \phi$ from 11/13
	\end{enumerate}
	\item Introduction
	\item Hyperbolic Geometry and the \poincare Disk model
	\begin{enumerate}[A.]
		\item Discussion of Euclid's postulates 
		\item Intuition -- ``saddle" space figure and discussion of interior angles of triangles
		\item Distance metric and line construction in the \poincare Disk model
	\end{enumerate}
	\item Reconstruction of (parts of) Guggenheimer's paper with coefficients in $\R$
	\begin{enumerate}[A.]
		\item Notation: matrices, linear fractional transformations, points, ends, poles/polars etc.
		\item Linear fractional transformations: algebra and corresponding figures
		\item Point reflections: algebra and corresponding figures
		\item Line reflections: algebra and corresponding figures
		\item Line reflections through an end: algebra and corresponding figures
	\end{enumerate}	
	\item Discussions of complexification of fields 
		\item Reconstruction of (parts of) Guggenheimer's paper with coefficients in $\R^{\C} = \C$
	\begin{enumerate}[A.]
		\item To be decided (based on what we discover while generating figures)
	\end{enumerate}	
	\item Conclusion

\end{enumerate}






\end{document}
